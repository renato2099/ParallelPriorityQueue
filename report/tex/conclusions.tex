\section{Conclusions}

Here you need to summarize what you did and why this is
important. {\em Do not take the abstract} and put it in the past
tense. Remember, now the reader has (hopefully) read the paper, so it
is a very different situation from the abstract. Try to highlight
important results and say the things you really want to get across
such as high-level statements (e.g., we believe that .... is the right
approach to .... Even though we only considered the DFT, the
.... technique should be applicable ....) You can also formulate next
steps if you want. Be brief


------------------
In spite of the fact that our implementation is lock-free, it may not be starvation-free. 
This is because there could be a thread A that when going through the lowest level 
of the SkipList searching for the next un-marked node (i.e. logically undeleted) is always 
outrun by some other thread B. This means that thread A can fail repeatedly if the others
threads always succeed.
Thread contention may happen if many threads try to logically delete a node (i.e. mark
a node). Only one thread will succeed and all the other unsuccessful ones will race to mark 
the next available node. Another problem could arise when some node physically deletes a
node that other threads are still using. This will lead into repeated \it{compareAndSet} failures.

The SkipList is a multilevel linked-list. The XeonPhi uses tag-directory caches with explicit updates. 
In our project this means that every every time a thread updates a node, the operation will also be copied 
into the remote caches. In spite of the fact that all CPUs can be updated at once, multiple writes
cause more traffic. And this is, of course, a more expensive operation on the XeonPhi.
