% IEEE standard conference template; to be used with:
%   spconf.sty  - LaTeX style file, and
%   IEEEbib.bst - IEEE bibliography style file.
% --------------------------------------------------------------------------

\documentclass[letterpaper]{article}
\usepackage{spconf,amsmath,amssymb,graphicx}

\usepackage{listings}

% Example definitions.
% --------------------
% nice symbols for real and complex numbers
\newcommand{\R}[0]{\mathbb{R}}
\newcommand{\C}[0]{\mathbb{C}}

% bold paragraph titles
\newcommand{\mypar}[1]{{\bf #1.}}

% Title.
% ------
\title{A Concurrent Priority Queue for the Xeon Phi}
%
% Single address.
% ---------------
\name{Markus P\"uschel\thanks{The author thanks Jelena Kovacevic. This paper
is a modified version of the template she used in her class.}} 
\address{Department of Computer Science\\ ETH Z\"urich\\Z\"urich, Switzerland}

% For example:
% ------------
%\address{School\\
%		 Department\\
%		 Address}
%
% Two addresses (uncomment and modify for two-address case).
% ----------------------------------------------------------
%\twoauthors
%  {A. Author-one, B. Author-two\sthanks{Thanks to XYZ agency for funding.}}
%		 {School A-B\\
%		 Department A-B\\
%		 Address A-B}
%  {C. Author-three, D. Author-four\sthanks{The fourth author performed the work
%		 while at ...}}
%		 {School C-D\\
%		 Department C-D\\
%		 Address C-D}
%

\begin{document}
%\ninept
%
\maketitle
%

%The hard page limit is 6 pages in this style. Do not reduce font size or use other tricks to squeeze. This pdf is formatted in the American letter format, so may look a bit strange when printed out.

\begin{abstract}
%Describe in concise words what you do, why you do it (not necessarily
%in this order), and the main result.  The abstract has to be
%self-contained and readable for a person in the general area. You
%should write the abstract last.

	\begin{itemize}
	  \item Something about priority queues e.g. use cases, importance.
	  \item Why are there different implementations e.g. parallel vs concurrent, cpu vs gpu
	  \item Our proposed method and why we thought it was a good idea
	  \item Experiences while porting it into a MIC architecture
	\end{itemize}

\end{abstract}

\section{Introduction}\label{sec:intro}

% complexity
% lock-free
% operations
% implementantions O(k)...
% data locality
% other implementations, tbb lock-based, complexity data locality

A priority queue is a data structure similar to a queue, where each element is associated with a priority key.
Thus, each element has a pair of value and key.
An element with high priority is served before an element with low priority.
The \textit{insert} and \textit{deleteMin} are the two basic operations.
A priority queue can be implemented, by using various data structures.
Different data structures provide different time and space complexity.
The most common data structure for the implementation of a priority queue is a heap.
It requires $\mathcal{O}(\log{}n)$ time complexity for both operations, and $\mathcal{O}(n)$ space complexity.

% kapoia paradeigmata efarmogon pou exoun priority queue
%Then explain that fast implementations are very hard and expensive to get (memory hierarchy, vector, parallel). 

The trend of processors technology has moved from single-core to multi-core.
Therefore, multiple threads may access the same priority queue at the same time in multithreading applications.
The consistency of a concurrent priority queue can be achieved using mutual exclusion.
However, the use of locks causes blocking and deadlocks, and limits scalability.

In this paper, we present a lock-free concurrent priority queue.
The underlying data structure is a skip list, which provides probabilistic balancing based on multiple levels.

%RELATED WORK

%Do not start the introduction with the abstract or a slightly modified version. It follows a possible structure of the introduction.  Note that the structure can be modified, but the content should be the same. Introduction and abstract should fill at most the first page, better less.

%\mypar{Motivation} The first task is to motivate what you do.  You can start general and zoom in one the specific problem you consider.  In the process you should have explained to the reader: what you are doing, why you are doing, why it is important (order is usually reversed).

%For example, if my result is the fastest DFT implementation ever, one could roughly go as follows. First explain why the DFT is important (used everywhere with a few examples) and why performance matters (large datasets, realtime). Then explain that fast implementations are very hard and expensive to get (memory hierarchy, vector, parallel). 

%Now you state what you do in this paper. In our example: presenting a DFT implementation that is faster for some sizes as all the other ones.

%\mypar{Related work} Next, you have to give a brief overview of related work. For a paper like this, anywhere between 2 and 8 references. Briefly explain what they do. In the end contrast to what you do to make now precisely clear what your contribution is.


\section{Background}\label{sec:background}

A priority queue is an abstract data structure which can be implemented using various data structures.
An efficient implementation can be achieved with a heap which is based on an array.
It requires $\mathcal{O}(\log{}n)$ time complexity for both operations, \textit{push} and \textit{pop}, and $\mathcal{O}(n)$ space complexity.
The TBB implementation that we use in this paper, is based on an array.
We also use the C++ standard libary (STD) implementation which is based on an array, but it is not thread safe.
The consistency of it, is achieved manually using mutual exclusion.
Other implementations of a concurrent priority queue may provide faster operations thanks to time and space complexity, and data locality.

Our lock-free implementation is based on a skip list \cite{Pugh:1990:SLP:78973.78977}  which provides probabilistic balancing.
The bottom level is an ordinary ordered linked list, and each of the other levels acts as a subset of the elements of the lower levels which assist in faster operations.
Each element of a level \textit{i} appears on the level \textit{i+1} with probability 0.5.
Therefore, the first level contains all the elements, the second one contains 50\% of them, the third one 25\% and so on.

A skip list requires more memory than other data structures (e.g a heap which is implemented using an array), since every node has multiple pointers to the next node.
However, it supports the \textit{pop} operation in $\Theta(1)$, since the elements are ordered and the element with the maximum priority is always located in the beginning.
The time complexity for the \textit{push} operation is $\mathcal{O}(\log{}n)$.


\section{Proposed Approach}\label{sec:approach}


\begin{itemize}
	\item Lock-based implementation
	\item Lock-free implementation
	\item Correctness tests with applications
	\item Explain advantages vs disadvantages of approach
\end{itemize}

We decided to base the implementation of our concurrent priority queue on a lockfree Skiplist. A lockfree Skiplist is a good choice for a concurrent datastructure since the atomic operations are very local to 2-3 nodes and thereby the probabilty of conflicts between threads is minimized. Since our final goal was a port on to the Xeon Phi which comes with 60 cores this would be a good fit.\\
The Skiplist was implemented using C++ atomics and its {\em comapre\_and\_exchange} operation.\\
In Figure XX we have an illustration of Skiplist which essentially is a linked list with a hierarchial tree-like index structure. The index creation is probabilistic but should asymptotically lead to a structure which enables traversing to any node in $O(\log n)$.\\
MAYBE EXPLAIN some operations\\
Like most lockfree data structures we had to deal with the problem {\em predecessor-deletion} [FIND BETTER WORD, see paper]. The problem is explained in figure XX.
A common resolution is to add a delete flag to each node for lazy deletion. But this means it is necessary to do CAS operation on the {\em next} pointer and the delete flag simulatenously. Multiple options are possible: {\em Atomic Markable Reference}, Double CAS, double-width CAS, Transactional-Memory. For our implementation we used the first option and implemented a {\em Atomic Markable Reference} by using the least significant bit as a deletion flag. Thereby we are able to use the general {\em compare\_and\_exchange} operation in C++11. Since our next pointers are always alligned to 64bytes [what is the architecture doing??]. In general no issues should arise when using the least significant bit as a deletion flag.\\
Apart from the common Skiplist methods: empty(), size(), find(), remove(), insert(). We added a pop() method necessary for a priority queue and pop(k) and insert(k) for batch processing.\\


%MAYBE give complete API somewhere
\begin{lstlisting}[language=C++,basicstyle=\tt\footnotesize,captionpos=b,caption=PPQ interface,morekeywords={*, size_t}]
template <class T, class Comp> class PPQ
{
	bool empty() const;
	size_t size() const;
	bool push(const T& data);
	size_t push(T data[], int k);
	bool remove(const T& data);
	bool pop_front(T& data);
	size_t pop_front(T data[], int k);
	bool contains(T data);
	void print();
};
\end{lstlisting}

\subsection{Correcteness verification}
\label{subsec:corr_ver}
We used two different applications for verifying the correctness of our algorithm, a loseless data compression algorithm and an algorithm to find the shortest path in a graph. 
%\mypar{Huffman coding algorithm}
We implemented the Huffman coding algorithm as an initial correcteness test for our data structure implementation. This algorithm consists in two main steps. The first step is to initialize a data structure which keeps all the items sorted by their frequency. Then in a second step, it iterates until the data structure has a single element. Every iteration consists of picking the two nodes with the lowest frequency/probability, creating a parent node out of them with the sum of the children's frequencies/probabilities, inserting this new node into the data structure and assigning code zero, or one to the children, and delete them from the data structure. We compared the output of a Huffman coding algorithm using a min-heap with an implementation using our priority queue. We only compared correctness between this two implementations as this loseless compression algorithm is based on iterating linearly through the elements of the underlying data structure and no concurrent access is done.
%\mypar{Shortest path algorithm}
A shortest path algorithm performs searches over a graph to determine the a path with the least number of hops between two nodes. Thus, we tested the correctness of our application against the shortest path algorithm implementation provided by Intel TBB. This implementation uses Intel TBB concurrent priority queue by default, and we adapted to also use our priority queue. We verified the correctness of both outputs. We didn't measure any performance over these algorithms as this might have implied some specific tuning for each of them. This falls out of the scope of this project. In section~\ref{sec:exp}, we do performance evaluation against Intel TBB concurrent priority queue with specific workloads.

\subsection{Memory allocation micro-benchmark}
We performed a microbenchmark to evaluate the memory allocation process in the XeonPhi. We did three different types of mememory allocation: First, we used the primites offered by the programming language, then we used compiler hints offered by the Intel Thread Building Blocks to use its scalable allocator implementation, and finally we allocated the memory needed sequentially using compiler hints too. These options are described below.

\mypar{C++ memory allocation primitives}
These are optimized for sequential case. They were designed with two main objectives: use efficiently memory space and minimize CPU overhead. However, these objectives do not target achieving good parallel performance. Modern hardware provides larger memory sizes, but also a bigger gap between CPU and memory speed. Thus, cache locality and avoidance of false sharing play a more important role.

\mypar{Intel TBB scalable allocator}
It uses \textit{thread-private heaps} to reduce the amount of code that needs synchronization and also false sharing. Each thread allocates its own copy of heap structures and accesses it via thread-specific data (TSD) using corresponding system APIs.
The allocator gets 1MB chunks from the operating system and divides them into 16K-byte aligned blocks. Then, it places these blocks initially in a global heap of free blocks. Memory requested is not returned to the operating system (but only in large allocation cases). Thus, it can make sure that memory is reused. Additional blocks are requested if a thread does not find free objects in the blocks of its own heap and there are no available blocks in the global heap~\cite{_thefoundations,Hudson:2006:MST:1133956.1133967}.
To use the scalable allocator offered by Intel TBB we had to link two libraries:
\begin{description}
	\item[-ltbb] To use the Intel TBB library.
	\item[-ltbbmalloc] To use the scalable allocator offered by the Intel TBB library.
\end{description}
 
\mypar{Sequential allocation with compiler hints}
Every node in our skip list contains an array of pointers to the next nodes depending on the level each node belongs to. This means that parts of this array might fall into different cache lines. Thus, our data structure can suffer more from cache misses and false sharing. We designed each skip list node to contain an empty array and allocating only the pointers that it will actually need (listing~\ref{lst:freenode}). In addition to that, the usage of Intel TBB allocator helps us ensuring that these nodes will be cache-aligned (listing~\ref{lst:fn_alloc}).

\begin{lstlisting}[language=C++,basicstyle=\tt\footnotesize,captionpos=b,caption=Lock free node structure,label=lst:freenode,morekeywords={*, size_t}]
template <typename T>  struct LockFreeNode
{	
	T data;
	int	level;
	AtomicRef<LockFreeNode>	next[0];
};
\end{lstlisting}

\begin{lstlisting}[language=C++,basicstyle=\tt\footnotesize,captionpos=b,caption=Memory allocation instruction for array of atomic references,label=lst:fn_alloc, morekeywords={*, size_t}]

scalable_malloc(
sizeof(LockFreeNode) 
+ 
((arrayLength + 1) * sizeof(AtomicRef<LockFreeNode>))
)
\end{lstlisting}

\mypar{Memory allocation microbenchmnark}
We designed our microbenchmark to allocate ten million skip list nodes while using 240 threads. We measured the time it takes to complete the allocation task on the XeonPhi. When using the regular memory allocation method, we experienced really high latency but when we used Intel TBB scalable allocator we obtained a 24, and 38 times improvemente for using TBB library, and a sequential allocation with Intel TBB allocator respectively. The results are displayed on figure~\ref{fig:mem_alloc}.


% explain data + graph 
\begin{figure}
	\centering
  	\includegraphics[scale=0.35]{../plots/mem_alloc/mem_alloc.pdf}
	\caption{Memory allocation runtime among a regular allocation, using compiler hints, and using sequential allocation with compiler hints}
	\label{fig:mem_alloc}
\end{figure}


\section{Experimental Results}\label{sec:exp}

\begin{itemize}
	\item CPU results
		\begin{itemize}
			\item Insert-only workload
			\item Pop-only workload
			\item Mix workload
		\end{itemize}
	\item Xeon Phi results
		\begin{itemize}
			\item Insert-only workload
			\item Pop-only workload
			\item Mix workload
		\end{itemize}
	\item Explanation on why we are not fast
\end{itemize}

\mypar{Operational intensity}
% recap of operational intensity
Operational intensity is defined as the ratio of the number of instructions executed to the number of memory accesses(look for citation?). If there exist many instructions per memory access, then the program is considered to have a high computational intensity i.e. compute bounded. On the other hand, if there are a small number of instructions are executed per memory access, then the program is considered to have a low computational intensity i.e. memory bounded.
% why we think it matters in our case
Our project goal was to design a simple, yet effective, priority queue. Thus, we expected to have an operational intensity dominated mainly by the number of memory accesses, and aimed to improve this. Having to move data around has a different impact on CPU architectures. We will describe and explain how our data structure behaves on Intel Haswell microarchitecture (Intel Core i7-4558U) and on Ivy Bridge microarchitecture (Intel Core i7-3820).
% Differences between these two microarchitectures
%TODO fix bib
The Intel Haswell microarchiteture is the successor of Ivy Bridge. They have several differences but they also share many commonalities. One of the biggest change is the memory hierarchy implemented on the Intel Haswell. The cache bandwidth doubled and its memory sytem can now perform two loads and one store per cycle. The Haswell's L1 load bandwidth is of 64 bytes/cycle, its L1 store bandwidth is of 32 bytes/cycle and also L2 bandwidth to L1 has doubled (from 32 bytes/cycle to 64 bytes/cycle). Other relevant improvements are the ones related to the Translation Look-aside Buffer (TLB) which in Haswell has access to 2M shared pages. The page entry also doubled in Haswell as well as the associativity; It went from a 4-way associative TLB in Ivy Bridge to a 8-way associative TLB in Haswell.

\begin{table}[h]
\begin{tabular}{|l|l|l|ll}
\cline{1-3}
\multicolumn{1}{|c|}{\textbf{Metric}} & \multicolumn{1}{c|}{\textbf{Ivy Bridge}} & \multicolumn{1}{c|}{\textbf{Haswell}} &  &  \\ \cline{1-3}
L1 Load Bandwidth                     & 32 Bytes/cycle                           & 64 Bytes/cycle                        &  &  \\ \cline{1-3}
L1 Store bandwidth                    & 16 Bytes/cycle                           & 32 Bytes/cycle                        &  &  \\ \cline{1-3}
L2 Bandwidth to L1                    & 32 Bytes/cycle                           & 64 Bytes/cycle                        &  &  \\ \cline{1-3}
L2 Unified TLB                        & 4K:512, 4-way                            & 4k+2M shared: 1024, 8-way             &  &  \\ \cline{1-3}
\end{tabular}
\end{table}

%~\cite{http://ijcsit.com/docs/Volume%204/vol4Issue3/ijcsit2013040321.pdf, http://www.agner.org/optimize/microarchitecture.pdf, http://web.eecs.utk.edu/courses/fall2013/cosc530/CS530Project_intel.pdf}
% explain data + graph + core architecture


\section{Conclusions}
\label{sec:con}
We presented a lock-free concurrent priority queue based on a skip list. The implementation was evaluated on a common x86 CPU and more importantly on a Xeon Phi MIC. The results show that our implementation is competitive in comparison to asses baseline implementations. On the MIC architecture, all tested implementations are below our expectations and are unable to take advantage of the hardware. Our lock-free implementation suffers mostly from the overhead due to the cache-consistency protocol.


%\section{Further comments}

%Here we provide some further tips.

%\mypar{Further general guidelines}

%\begin{itemize}
%\item For short papers, to save space, I use paragraph titles instead of
%subsections, as shown in the introduction.
%
%\item It is generally a good idea to break sections into such smaller
%units for readability and since it helps you to (visually) structure the story.
%
%\item The above section titles should be adapted to more precisely
%reflect what you do.
%
%\item Each section should be started with a very
%short summary of what the reader can expect in this section. Nothing
%more awkward as when the story starts and one does not know what the
%direction is or the goal.
%
%\item Make sure you define every acronym you use, no matter how
%convinced you are the reader knows it.
%
%\item Always spell-check before you submit (to me in this case).
%
%\item Be picky. When writing a paper you should always strive for very
%high quality. Many people may read it and the quality makes a big difference.
%In this class, the quality is part of the grade.
%
%\item Books helping you to write better: \cite{Higham:98} and \cite{Strunk:00}.
%
%\item Conversion to pdf (latex users only): 
%
%dvips -o conference.ps -t letter -Ppdf -G0 conference.dvi
%
%and then
%
%ps2pdf conference.ps
%\end{itemize}
%
%\mypar{Graphics} For plots that are not images {\em never} generate
%jpeg, gif, bmp, tif. Use eps, which means encapsulate postscript. It
%scalable since it is a vector graphic description of your graph. E.g.,
%from Matlab, you can export to eps.
%
%Here is an example of how to get a plot into latex
%(Fig.~\ref{fftperf}). Note that in this plot the text should be
%a little bit larger. In particular, the labels are too small!
%
%\begin{figure}\centering
%  \includegraphics[scale=0.33]{dft-performance.eps}
%  \caption{Performance of four single precision implementations of the
%  discrete Fourier transform. The operations count is roughly the
%  same. {\em The labels in this plot are too small.}\label{fftperf}}
%\end{figure}



% References should be produced using the bibtex program from suitable
% BiBTeX files (here: bibl_conf). The IEEEbib.bst bibliography
% style file from IEEE produces unsorted bibliography list.
% -------------------------------------------------------------------------
\bibliographystyle{IEEEbib}
\bibliography{bibl_conf}

\end{document}

