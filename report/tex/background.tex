\section{Background}
\label{sec:background}

The implementation of a priority queue can be achieved using different underlying data structures.
A common implementation is a binary-heap which is based on an array. It requires $\mathcal{O}(\log{}n)$ time complexity for \textit{push} and \textit{pop} operations, and has a space complexity of $\mathcal{O}(n)$ .

Two baseline implementations are used for comparison with our work.
The \textit{priority\_queue} from the C++ standard library (STD)[REF] and the \textit{concurrent\_priotiy\_queue} implemented in the Intel TBB library[REF].
The TBB implementation is based on concurrent vectors while the STD implementation uses vectors (by default) as the underlying data structure.
For the experimental evaluation presented in section \ref{sec:exp}, the STD implementation was made thread-safe, by adding a single mutex which serializes every access to the data structure.

Our lock-free implementation is based on a skip list \cite{Pugh:1990:SLP:78973.78977}.
As shown in figure XYZ, a skip list is a multi-level list where the list on the bottom level is equal to a single-linked list.
The upper level lists have fewer nodes, in the sense that some nodes are skipped, thereby reducing the time to traverse them.
% TODO Over how many levels a certain element spawns is determined upon insertion.
Every node is pushed into the bottom level and reaches the next level with a probability of 0.5.
A node occurs in level 2 with probability 0.5, in level 3 with probability 0.25, and so on.
This leads to an index which is asymptotically equal to a binary tree.
Therefore, every node on the bottom level can be accessed in $\mathcal{O}(\log{}n)$.
Since the nodes are sorted, the \textit{pop} operation which removes the node with the highest priority, takes only $\mathcal{O}(1)$.

On average, every node has two pointers to other nodes, and the same size as a node in a binary tree or heap.
Consequently, the memory consumption of a skip list is similar.

A lock-free skip list is a good choice for a concurrent priority queue, since the atomic operations are very local to two, three nodes and the probabilty of conflicts between threads is minimized.
That helps porting onto Xeon Phi which has 60 cores, and is suitable for running tens of threads.
