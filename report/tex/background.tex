\section{Background}\label{sec:background}

A priority queue is an abstract data structure which can be implemented using various data structures.
An efficient implementation can be achieved with a heap which is based on an array.
It requires $\mathcal{O}(\log{}n)$ time complexity for both operations, \textit{push} and \textit{pop}, and $\mathcal{O}(n)$ space complexity.
The TBB implementation that we use in this paper, is based on an array.
We also use the C++ standard libary (STD) implementation which is based on an array, but it is not thread safe.
The consistency of it, is achieved manually using mutual exclusion.
Other implementations of a concurrent priority queue may provide faster operations thanks to time and space complexity, and data locality.

Our lock-free implementation is based on a skip list \cite{Pugh:1990:SLP:78973.78977}  which provides probabilistic balancing.
The bottom level is an ordinary ordered linked list, and each of the other levels acts as a subset of the elements of the lower levels which assist in faster operations.
Each element of a level \textit{i} appears on the level \textit{i+1} with probability 0.5.
Therefore, the first level contains all the elements, the second one contains 50\% of them, the third one 25\% and so on.

A skip list requires more memory than other data structures (e.g a heap which is implemented using an array), since every node has multiple pointers to the next node.
However, it supports the \textit{pop} operation in $\Theta(1)$, since the elements are ordered and the element with the maximum priority is always located in the beginning.
The time complexity for the \textit{push} operation is $\mathcal{O}(\log{}n)$.
